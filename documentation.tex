\NewDocumentCommand{\rsCodeAux}{ s m }{%
   \texttt{\IfBooleanTF{#1}{\textbackslash}\,#2}%
}

\NewDocumentCommand{\rsOptionsAux}{ s O{options} }{%
   \texttt{\IfBooleanTF{#1}{\textbackslash}<\textcolor{red!75}{#2}>}%
}

\NewDocumentCommand{\rsArgAux}{ s O{clist} }{%
   \texttt{\IfBooleanTF{#1}{\textbackslash}<\textcolor{blue!75}{#2}>}%
}

\NewDocumentCommand{\rsTypeAux}{ s O{cmd} }{%
   \texttt{\IfBooleanTF{#1}{\textbackslash}\textcolor{ForestGreen!75}{#2}}%
}


\newpage

\tableofcontents

\newpage


\section{Морфемный анализ слов} 

\subsection{Приставка}

\rsTypeAux:\rsCodeAux{rsPrefix[\rsOptionsAux]\{\rsArgAux\}} \\
\hspace*{1cm} \rsOptionsAux: \rsCodeAux{color:bool=false, phantom:bool=false}. 

\begingroup
\renewcommand{\arraystretch}{1.125}
\begin{table}[h!]
    \centering
    \begin{tabular}{|l|l|}
        \hline
        \rsCodeAux*{rsPrefix{\{\}}} & \rsPrefix{} \\
        \rsCodeAux*{rsPrefix{\{приставка\}}} & \rsPrefix{приставка} \\
        \rsCodeAux*{rsPrefix{\{при, став, ка\}}} & \rsPrefix{при, став, ка} \\
        \rsCodeAux*{rsPrefix[color]{\{при, став, ка\}}} & \rsPrefix[color]{при, став, ка} \\
        \rsCodeAux*{rsPrefix[phantom]{\{при, став, ка\}}} & \rsPrefix[phantom]{при, став, ка} \\
        \rsCodeAux*{rsPrefix[color, phantom]{\{при, став, ка\}}} & \rsPrefix[color, phantom]{при, став, ка} \\
        \hline
    \end{tabular}
    \caption{Использование команды приставки.}
\end{table}
\endgroup




\subsection{Корень}

\rsTypeAux:\rsCodeAux{rsRoot[\rsOptionsAux]\{\rsArgAux\}} \\
\hspace*{1cm} \rsOptionsAux: \rsCodeAux{color:bool=false, phantom:bool=false}.

\begingroup
\renewcommand{\arraystretch}{1.125}
\begin{table}[h!]
    \centering
    \begin{tabular}{|l|l|}
        \hline
        \rsCodeAux*{rsRoot{\{\}}} & \rsRoot{} \\
        \rsCodeAux*{rsRoot{\{корень\}}} & \rsRoot{корень} \\
        \rsCodeAux*{rsRoot{\{кор, ень\}}} & \rsRoot{кор, ень} \\
        \rsCodeAux*{rsRoot[color]{\{кор, ень\}}} & \rsRoot[color]{кор, ень} \\
        \rsCodeAux*{rsRoot[phantom]{\{кор, ень\}}} & \rsRoot[phantom]{кор, ень} \\
        \rsCodeAux*{rsRoot[color, phantom]{\{кор, ень\}}} & \rsRoot[color, phantom]{кор, ень} \\
        \hline
    \end{tabular}
    \caption{Использование команды корня.}
\end{table} 
\endgroup




\subsection{Суффикс}

\rsTypeAux:\rsCodeAux{rsSuffix[\rsOptionsAux]\{\rsArgAux\}} \\
\hspace*{1cm} \rsOptionsAux: \rsCodeAux{color:bool=false, phantom:bool=false}.

\begingroup
\renewcommand{\arraystretch}{1.125}
\begin{table}[h!]
    \centering
    \begin{tabular}{|l|l|}
        \hline
        \rsCodeAux*{rsSuffix{\{\}}} & \rsSuffix{} \\
        \rsCodeAux*{rsSuffix{\{суффикс\}}} & \rsSuffix{суффикс} \\
        \rsCodeAux*{rsSuffix{\{суф, фикс\}}} & \rsSuffix{суф, фикс} \\
        \rsCodeAux*{rsSuffix[color]{\{суф, фикс\}}} & \rsSuffix[color]{суф, фикс} \\
        \rsCodeAux*{rsSuffix[phantom]{\{суф, фикс\}}} & \rsSuffix[phantom]{суф, фикс} \\
        \rsCodeAux*{rsSuffix[color, phantom]{\{суф, фикс\}}} & \rsSuffix[color, phantom]{суф, фикс} \\
        \hline
    \end{tabular}
    \caption{Использование команды суффикса.}
\end{table}
\endgroup




\subsection{Окончание}

\rsTypeAux:\rsCodeAux{rsEnding[\rsOptionsAux]\{\rsArgAux[tl]\}} \\
\hspace*{1cm} \rsOptionsAux: \rsCodeAux{color:bool=false, phantom:bool=false}.

\begingroup
\renewcommand{\arraystretch}{1.125}
\begin{table}[h!]
    \centering
    \begin{tabular}{|l|l|}
        \hline
        \rsCodeAux*{rsEnding{\{\}}} & \rsEnding{} \\
        \rsCodeAux*{rsEnding{\{окончание\}}} & \rsEnding{окончание} \\
        \rsCodeAux*{rsEnding[color]{\{окончание\}}} & \rsEnding[color]{окончание} \\
        \rsCodeAux*{rsEnding[phantom]{\{окончание\}}} & \rsEnding[phantom]{окончание} \\
        \rsCodeAux*{rsEnding[color, phantom]{\{окончание\}}} & \rsEnding[color, phantom]{окончание} \\
        \hline
    \end{tabular}
    \caption{Использование команды окончания.}
\end{table}
\endgroup




\subsection{Постфикc}

\rsTypeAux:\rsCodeAux{rsPostfix[\rsOptionsAux]\{\rsArgAux\}} \\
\hspace*{1cm} \rsOptionsAux: \rsCodeAux{color:bool=false, phantom:bool=false}.

\begingroup
\renewcommand{\arraystretch}{1.125}
\begin{table}[h!]
    \centering
    \begin{tabular}{|l|l|}
        \hline
        \rsCodeAux*{rsPostfix{\{\}}} & \rsPostfix{} \\
        \rsCodeAux*{rsPostfix{\{постфикс\}}} & \rsPostfix{постфикс} \\
        \rsCodeAux*{rsPostfix{\{пост, фикс\}}} & \rsPostfix{пост, фикс} \\
        \rsCodeAux*{rsPostfix[color]{\{пост, фикс\}}} & \rsPostfix[color]{пост, фикс} \\
        \rsCodeAux*{rsPostfix[phantom]{\{пост, фикс\}}} & \rsPostfix[phantom]{пост, фикс} \\
        \rsCodeAux*{rsPostfix[color, phantom]{\{пост, фикс\}}} & \rsPostfix[color, phantom]{пост, фикс} \\
        \hline
    \end{tabular}
    \caption{Использование команды постфикса.}
\end{table}
\endgroup




\subsection{Основа}

\rsTypeAux:\rsCodeAux{rsBase[\rsOptionsAux]\{\rsArgAux[tl]\}} \\
\hspace*{1cm} \rsOptionsAux: \rsCodeAux{color:bool=false, right:bool=false, left:bool=false}.

\begingroup
\renewcommand{\arraystretch}{1.125}
\begin{table}[h!]
    \centering
    \begin{tabular}{|l|l|}
        \hline
        \rsCodeAux*{rsBase{\{основа\}}} & \rsBase{основа} \\
        \rsCodeAux*{rsBase[color]{\{основа\}}} & \rsBase[color]{основа} \\
        \rsCodeAux*{rsBase[left]{\{основа\}}} & \rsBase[left]{основа} \\
        \rsCodeAux*{rsBase[color, right]{\{основа\}}} & \rsBase[color, right]{основа} \\
        \hline
    \end{tabular}
    \caption{Использование команды основы.}
\end{table}
\endgroup




\subsection{Разбор слова с непрерывной основной}

{\small \rsTypeAux:\rsCodeAux{rsMorphemicAnalysis[\rsOptionsAux]\{\rsArgAux\}\{\rsArgAux\}\{\rsArgAux\}\{\rsArgAux[tl]\}\{\rsArgAux\}}} \\
\hspace*{1cm} \rsOptionsAux: \rsCodeAux{color:bool=false, phantom:bool=false}.

\begingroup
\renewcommand{\arraystretch}{1.125}
\begin{table}[h!]
    \centering
    \begin{tabular}{|l|l|}
        \hline
        {\small \rsCodeAux*{rsMorphemicAnalysis\{бес, при\}\{дан\}\{н, ниц\}\{а\}\{\}}} & \rsMorphemicAnalysis{бес, при}{дан}{н, ниц}{а}{} \\
        {\small \rsCodeAux*{rsMorphemicAnalysis\{из\}\{маз\}\{а, л\}\{\}\{ся\}}} & \rsMorphemicAnalysis{из}{маз}{а, л}{}{ся} \\
        {\small \rsCodeAux*{rsMorphemicAnalysis[phantom]\{из\}\{маз\}\{а, л\}\{\}\{ся\}}} & \rsMorphemicAnalysis[phantom]{из}{маз}{а, л}{}{ся} \\
        {\small \rsCodeAux*{rsMorphemicAnalysis[color]\{вне\}\{штат\}\{н\}\{ый\}\{\}}} & \rsMorphemicAnalysis[color]{вне}{штат}{н}{ый}{} \\
        \hline
    \end{tabular}
    \caption{Использование команды разбора слова.}
\end{table}
\endgroup




\section{Части речи}


\section{Синтаксический разбор предложений}

\subsection{Подлежащее}

\subsection{Сказуемое}

\subsection{Дополнение}

\subsection{Определение}

\subsection{Обстоятельство}



\section{Прочее}

\subsection{Окружение \rsCodeAux{rslingu}}

\noindent\rsTypeAux[env]:\rsCodeAux{rslingu[\rsOptionsAux]} \\
\hspace*{1cm} \rsOptionsAux: \rsCodeAux{color:bool=false, phantom:bool=false}.

Иногда может возникать необходимость, например, морфемного разбора слов с <<разрывной>> основой. Для таких случаев нет специально
определённых команд, подобно команде \rsCodeAux*{rsMorphemicAnalysis}, так что единственный способ отобразить такие слова --- это
последовательное использование команд для каждой из морфем. При передаче параметров \rsCodeAux{phantom} и \rsCodeAux{color} в каждую из
команд возникает многократное дублирование кода, что ухудшает его читаемость.



Решить эту проблему призвано окружение \rsCodeAux{rslingu}, которое указании какого-либо дополнительного аргумента, <<активирует>> его для всех команд, принимающий данный аргумент, внутри окружения.
