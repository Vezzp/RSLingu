\documentclass[14pt, dvipsnames]{extarticle}

\usepackage[bottom=2cm,
            left=2cm,
            right=2cm,
            top=2cm]{geometry}

\linespread{1.5}

\usepackage{ifxetex}
\ifxetex
    \usepackage[cm-default]{fontspec}
    \defaultfontfeatures{Ligatures=TeX}
    \usepackage{polyglossia}
    \setdefaultlanguage{russian}
    \setmainfont{Times New Roman}
    \newfontfamily\cyrillicfont{Times New Roman}
    \setmonofont{CMU Typewriter Text}
\else
   \usepackage[utf8]{inputenc}
   \usepackage[russian]{babel}
\fi

\usepackage{RSLingu/rslingu}

\usepackage{multicol}
\usepackage{pythontex}

\begin{document}

Начинайте набирать текст.


    % \NewDocumentCommand{\rsSubject}{ s m }{%
    %     \rsUnderlineSubject
    %         [\IfBooleanTF{#1}{RSSubjectColor}{}]
    %         {\ProcessMap{\textoverset}{#2}[ ]}%
    % }
    
    
% Syntax key-val arguments.

    
    

% \ExplSyntaxOn

% \NewDocumentCommand{\rsSubject}{ O{} m }{%
%     \keys_set_known:nn { rs/syntax/keys } {color=false, phantom=false, width, #1}

%     \tikz[baseline=(Root.base)]{
%         \l__rs_syntax_choose_text:p  { #2 }
%         \l__rs_choose_color:n { RSSubjectColor }
%         \draw[\l__rs_color, RSUline]
%             ([yshift=-\RSUlineYShift pt]Root.south~west) --
%             ([yshift=-\RSUlineYShift pt]Root.south~east);
%     }
% }

% \ExplSyntaxOff



% \ExplSyntaxOn

% \NewDocumentCommand{\rsPredicate}{ O{} m }{%
%     \keys_set_known:nn { rs/syntax/keys } {color=false, phantom=false, width, #1}

%     \tikz[baseline=(Root.base)]{
%         \l__rs_syntax_choose_text:p  { #2 }
%         \l__rs_choose_color:n { RSPredicateColor }
%         \draw[\l__rs_color, RSUline]
%             ([yshift=-\RSUlineYShift pt]Root.south~west) --
%             ([yshift=-\RSUlineYShift pt]Root.south~east);
%         \draw[\l__rs_color, RSUline]
%             ([yshift=-\RSUlineYShift - 2 pt]Root.south~west) --
%             ([yshift=-\RSUlineYShift - 2 pt]Root.south~east);
%     }
% }

% \ExplSyntaxOff




% \ExplSyntaxOn

% \NewDocumentCommand{\rsAdverbarial}{ O{} m }{%
%     \keys_set_known:nn { rs/syntax/keys } {color=false, phantom=false, width, #1}

%     \tikz[baseline=(Root.base)]{
%         \l__rs_syntax_choose_text:p  { #2 }
%         \l__rs_choose_color:n { RSAdverbarialColor }
%         \draw[
%             \l__rs_color,
%             RSUline,
%             dash~pattern={on 5pt off 2pt on 1.5pt off 2pt},
%         ]
%             ([yshift=-\RSUlineYShift pt]Root.south~west) --
%             ([yshift=-\RSUlineYShift pt]Root.south~east);
%     }
% }

% \ExplSyntaxOff




% \ExplSyntaxOn

% \NewDocumentCommand{\rsAttribute}{ O{} m }{%
%     \keys_set_known:nn { rs/syntax/keys } {color=false, phantom=false, width, #1}

%     \tikz[baseline=(Root.base)]{
%         \l__rs_syntax_choose_text:p  { #2 }
%         \l__rs_choose_color:n { RSAttributeColor }
%         \draw[
%             \l__rs_color,
%             RSUline, 
%             decorate,
%             decoration={
%                 complete~sines,
%                 segment~length=2.75pt,
%                 amplitude=1.75,
%                 mirror,
%                 start~up,
%                 end~down
%             }
%         ]
%             ([yshift=-\RSUlineYShift pt]Root.south~west) --
%             ([yshift=-\RSUlineYShift pt]Root.south~east);
%     }
% }

% \ExplSyntaxOff




% \rsSubject[color]{подлежащее=сущ.}
% \rsPredicate[color, phantom]{сказуемое=глаг. + п.в. + м.р., сказуемое=глаг. + п.в.}
% \rsAdverbarial[color, phantom]{обстоятельство=сущ.}
% \rsAttribute[color]{определение}


\newpage 

\thispagestyle{empty}

\vspace*{7cm}
\begin{center}
    { \huge \rsCodeAux{RSLingu}. } \\
    { \Large \LaTeX-пакет для школьной лингвистики. }\\
\end{center}


\vspace*{7cm}
\begin{flushright}
    Автор
\end{flushright}


\newpage

\tableofcontents

\newpage


\chapter[Изучаем \LaTeX]{Изучаем \LaTeX}


\section{Где набираться знаний?}

\begin{itemize}
    \setlength{\itemsep}{0pt}
    \item В книге по \href{https://www.mccme.ru/free-books/llang/newllang.pdf}{ссылке}.
    \item В справочном практическом пособии по \href{http://www.ccas.ru/voron/download/voron05latex.pdf}{ссылке}.
    \item На бесплатном онлайн-курсе на Coursera по ссылке \href{https://ru.coursera.org/learn/latex}{ссылке}.
\end{itemize}


\section{Где оттачивать мастерство и работать?}

\begin{itemize}
    \setlength{\itemsep}{0pt}
    \item В интернете на \href{overleaf.com}{overleaf.com}.
    \item На персональном компьютере, установив программу MiKTeX по \href{https://miktex.org/download}{ссылке}.
\end{itemize}


\section{Ещё полезности?}

\begin{itemize}
    \setlength{\itemsep}{0pt}
    \item <<Исчерпывающий список символов>> по \href{http://tug.ctan.org/info/symbols/comprehensive/symbols-a4.pdf}{ссылке}.
    \item Репозиторий существующих пактов \href{ctan.org}{ctan.org}.
\end{itemize}



\chapter{Пакет \rsCodeAux{rslingu} для школьной лингвистики}


\section{Условные обозначения}

Формально описание команды или окружения будет даваться следующим образом, например:
\begin{tcolorbox}
    \small
    \rsTypeAux:\rsCodeAux{rsPrefix[\rsOptionsAux]\{\rsArgAux\}} \\
    \hspace*{1cm} \rsOptionsAux: \rsCodeAux{color:bool=false, phantom:bool=false}. 
\end{tcolorbox}

Разберём первую строчку, которая содержит сигнатуру объекта-модификатор текста:
\begin{center}
    \rsTypeAux:\rsCodeAux{rsPrefix[\rsOptionsAux]\{\rsArgAux\}}
\end{center}
Сначала идёт обозначения типа объекта-модификатора текста, он может быть либо \rsTypeAux\space --- для команды --- или \rsTypeAux[env] --- для окружения. Далее идёт \rsCodeAux{:}, смысловой нагрузки он несёт --- это просто разделитель для лучшей читаемости. Затем имя, как в данном примере, команды --- \rsCodeAux{rsPrefix}. 

После следуют аргументы, которые принимает команда или окружение. Обязательные аргументы пишутся в фигурных скобках; необязательные --- в квадратных. Указание \rsOptionsAux\space в квадратных скобках означает, что 




\section{Морфемный анализ слов} 

\subsection{Приставка}

\begin{tcolorbox}
    \small
    \rsTypeAux:\rsCodeAux{rsPrefix[\rsOptionsAux]\{\rsArgAux\}} \\
    \hspace*{1cm} \rsOptionsAux: \rsCodeAux{color:bool=false, phantom:bool=false}. 
\end{tcolorbox}

\begingroup
\renewcommand{\arraystretch}{1.125}
\begin{table}[h!]
    \centering
    \begin{tabular}{|l|l|}
        \hline
        \rsCodeAux*{rsPrefix{\{\}}} & \rsPrefix{} \\
        \rsCodeAux*{rsPrefix{\{приставка\}}} & \rsPrefix{приставка} \\
        \rsCodeAux*{rsPrefix{\{при, став, ка\}}} & \rsPrefix{при, став, ка} \\
        \rsCodeAux*{rsPrefix[color]{\{при, став, ка\}}} & \rsPrefix[color]{при, став, ка} \\
        \rsCodeAux*{rsPrefix[phantom]{\{при, став, ка\}}} & \rsPrefix[phantom]{при, став, ка} \\
        \rsCodeAux*{rsPrefix[color, phantom]{\{при, став, ка\}}} & \rsPrefix[color, phantom]{при, став, ка} \\
        \hline
    \end{tabular}
    \caption{Использование команды приставки.}
\end{table}
\endgroup




\subsection{Корень}

\begin{tcolorbox}
    \small
    \rsTypeAux:\rsCodeAux{rsRoot[\rsOptionsAux]\{\rsArgAux\}} \\
    \hspace*{1cm} \rsOptionsAux: \rsCodeAux{color:bool=false, phantom:bool=false}.
\end{tcolorbox}

\begingroup
\renewcommand{\arraystretch}{1.125}
\begin{table}[h!]
    \centering
    \begin{tabular}{|l|l|}
        \hline
        \rsCodeAux*{rsRoot{\{\}}} & \rsRoot{} \\
        \rsCodeAux*{rsRoot{\{корень\}}} & \rsRoot{корень} \\
        \rsCodeAux*{rsRoot{\{кор, ень\}}} & \rsRoot{кор, ень} \\
        \rsCodeAux*{rsRoot[color]{\{кор, ень\}}} & \rsRoot[color]{кор, ень} \\
        \rsCodeAux*{rsRoot[phantom]{\{кор, ень\}}} & \rsRoot[phantom]{кор, ень} \\
        \rsCodeAux*{rsRoot[color, phantom]{\{кор, ень\}}} & \rsRoot[color, phantom]{кор, ень} \\
        \hline
    \end{tabular}
    \caption{Использование команды корня.}
\end{table} 
\endgroup




\subsection{Суффикс}
    
\begin{tcolorbox}
    \small
    \rsTypeAux:\rsCodeAux{rsSuffix[\rsOptionsAux]\{\rsArgAux\}} \\
    \hspace*{1cm} \rsOptionsAux: \rsCodeAux{color:bool=false, phantom:bool=false}.
\end{tcolorbox}    

\begingroup
\renewcommand{\arraystretch}{1.125}
\begin{table}[h!]
    \centering
    \begin{tabular}{|l|l|}
        \hline
        \rsCodeAux*{rsSuffix{\{\}}} & \rsSuffix{} \\
        \rsCodeAux*{rsSuffix{\{суффикс\}}} & \rsSuffix{суффикс} \\
        \rsCodeAux*{rsSuffix{\{суф, фикс\}}} & \rsSuffix{суф, фикс} \\
        \rsCodeAux*{rsSuffix[color]{\{суф, фикс\}}} & \rsSuffix[color]{суф, фикс} \\
        \rsCodeAux*{rsSuffix[phantom]{\{суф, фикс\}}} & \rsSuffix[phantom]{суф, фикс} \\
        \rsCodeAux*{rsSuffix[color, phantom]{\{суф, фикс\}}} & \rsSuffix[color, phantom]{суф, фикс} \\
        \hline
    \end{tabular}
    \caption{Использование команды суффикса.}
\end{table}
\endgroup




\subsection{Окончание}

\begin{tcolorbox}
    \small
    \rsTypeAux:\rsCodeAux{rsEnding[\rsOptionsAux]\{\rsArgAux[tl]\}} \\
    \hspace*{1cm} \rsOptionsAux: \rsCodeAux{color:bool=false, phantom:bool=false}.
\end{tcolorbox}

\begingroup
\renewcommand{\arraystretch}{1.125}
\begin{table}[h!]
    \centering
    \begin{tabular}{|l|l|}
        \hline
        \rsCodeAux*{rsEnding{\{\}}} & \rsEnding{} \\
        \rsCodeAux*{rsEnding{\{окончание\}}} & \rsEnding{окончание} \\
        \rsCodeAux*{rsEnding[color]{\{окончание\}}} & \rsEnding[color]{окончание} \\
        \rsCodeAux*{rsEnding[phantom]{\{окончание\}}} & \rsEnding[phantom]{окончание} \\
        \rsCodeAux*{rsEnding[color, phantom]{\{окончание\}}} & \rsEnding[color, phantom]{окончание} \\
        \hline
    \end{tabular}
    \caption{Использование команды окончания.}
\end{table}
\endgroup




\subsection{Постфикc}

\begin{tcolorbox}
    \small
    \rsTypeAux:\rsCodeAux{rsPostfix[\rsOptionsAux]\{\rsArgAux\}} \\
    \hspace*{1cm} \rsOptionsAux: \rsCodeAux{color:bool=false, phantom:bool=false}.
\end{tcolorbox}

\begingroup
\renewcommand{\arraystretch}{1.125}
\begin{table}[h!]
    \centering
    \begin{tabular}{|l|l|}
        \hline
        \rsCodeAux*{rsPostfix{\{\}}} & \rsPostfix{} \\
        \rsCodeAux*{rsPostfix{\{постфикс\}}} & \rsPostfix{постфикс} \\
        \rsCodeAux*{rsPostfix{\{пост, фикс\}}} & \rsPostfix{пост, фикс} \\
        \rsCodeAux*{rsPostfix[color]{\{пост, фикс\}}} & \rsPostfix[color]{пост, фикс} \\
        \rsCodeAux*{rsPostfix[phantom]{\{пост, фикс\}}} & \rsPostfix[phantom]{пост, фикс} \\
        \rsCodeAux*{rsPostfix[color, phantom]{\{пост, фикс\}}} & \rsPostfix[color, phantom]{пост, фикс} \\
        \hline
    \end{tabular}
    \caption{Использование команды постфикса.}
\end{table}
\endgroup




\subsection{Основа}

\begin{tcolorbox}
    \small
    \rsTypeAux:\rsCodeAux{rsBase[\rsOptionsAux]\{\rsArgAux[tl]\}} \\
    \hspace*{1cm} \rsOptionsAux: \rsCodeAux{color:bool=false, right:bool=false, left:bool=false}.
\end{tcolorbox}

\begingroup
\renewcommand{\arraystretch}{1.125}
\begin{table}[h!]
    \centering
    \begin{tabular}{|l|l|}
        \hline
        \rsCodeAux*{rsBase{\{основа\}}} & \rsBase{основа} \\
        \rsCodeAux*{rsBase[color]{\{основа\}}} & \rsBase[color]{основа} \\
        \rsCodeAux*{rsBase[left]{\{основа\}}} & \rsBase[left]{основа} \\
        \rsCodeAux*{rsBase[color, right]{\{основа\}}} & \rsBase[color, right]{основа} \\
        \hline
    \end{tabular}
    \caption{Использование команды основы.}
\end{table}
\endgroup




\subsection{Разбор слова с непрерывной основной}

\begin{tcolorbox}
    \small
    \rsTypeAux:\rsCodeAux{rsMorphemicAnalysis[\rsOptionsAux]\{\rsArgAux\}\{\rsArgAux\}\{\rsArgAux\}\{\rsArgAux[tl]\}\{\rsArgAux\}} \\
    \hspace*{1cm} \rsOptionsAux: \rsCodeAux{color:bool=false, phantom:bool=false}.
\end{tcolorbox}    

\begingroup
\renewcommand{\arraystretch}{1.125}
\begin{table}[h!]
    \centering
    \begin{tabular}{|l|l|}
        \hline
        {\small \rsCodeAux*{rsMorphemicAnalysis\{бес, при\}\{дан\}\{н, ниц\}\{а\}\{\}}} & \rsMorphemicAnalysis{бес, при}{дан}{н, ниц}{а}{} \\
        {\small \rsCodeAux*{rsMorphemicAnalysis\{из\}\{маз\}\{а, л\}\{\}\{ся\}}} & \rsMorphemicAnalysis{из}{маз}{а, л}{}{ся} \\
        {\small \rsCodeAux*{rsMorphemicAnalysis[phantom]\{из\}\{маз\}\{а, л\}\{\}\{ся\}}} & \rsMorphemicAnalysis[phantom]{из}{маз}{а, л}{}{ся} \\
        {\small \rsCodeAux*{rsMorphemicAnalysis[color]\{вне\}\{штат\}\{н\}\{ый\}\{\}}} & \rsMorphemicAnalysis[color]{вне}{штат}{н}{ый}{} \\
        \hline
    \end{tabular}
    \caption{Использование команды разбора слова.}
\end{table}
\endgroup




\section{Части речи}


\section{Синтаксический разбор предложений}

\subsection{Подлежащее}

\subsection{Сказуемое}

\subsection{Дополнение}

\subsection{Определение}

\subsection{Обстоятельство}



\section{Прочее}

\subsection{Окружение \rsCodeAux{rslingu}}

\begin{tcolorbox}
    \noindent\rsTypeAux[env]:\rsCodeAux{rslingu[\rsOptionsAux]} \\
    \hspace*{1cm} \rsOptionsAux: \rsCodeAux{color:bool=false, phantom:bool=false}.
\end{tcolorbox}

Иногда может возникать необходимость, например, морфемного разбора слов с <<разрывной>> основой. Для таких случаев нет специально
определённых команд, подобно команде \rsCodeAux*{rsMorphemicAnalysis}, так что единственный способ отобразить такие слова --- это
последовательное использование команд для каждой из морфем. При передаче параметров \rsCodeAux{phantom} и \rsCodeAux{color} в каждую из
команд возникает многократное дублирование кода, что ухудшает его читаемость.



Решить эту проблему призвано окружение \rsCodeAux{rslingu}, которое указании какого-либо дополнительного аргумента, <<активирует>> его для всех команд, принимающий данный аргумент, внутри окружения.

Как водится, для более объемлющего понимания лучше один раз увидеть, чем читать <<сухой>> текст <<документации>>.




\end{document}