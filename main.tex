\documentclass[17pt, dvipsnames]{extarticle}

\usepackage[bottom=2cm,
            left=2cm,
            right=2cm,
            top=2cm]{geometry}

\linespread{1.5}

\usepackage{ifxetex}
\ifxetex
    \usepackage[cm-default]{fontspec}
    \defaultfontfeatures{Ligatures=TeX}
    \usepackage{polyglossia}
    \setdefaultlanguage{russian}
    \setmainfont{Times New Roman}
    \newfontfamily\cyrillicfont{Times New Roman}[Script=Cyrillic]
\else
   \usepackage[utf8]{inputenc}
   \usepackage[russian]{babel}
\fi

\usepackage{RSLingu/rslingu}


\begin{document}

Начинайте набирать текст.

% \rsAttribute*{привет}
% \rsObject*{привет}
% \rsSubject*{пивет}
% \rsAdverbarial*{привет}
% \rsPredicate*{привет}


% \rsAttribute{привет}
% \rsObject{привет}
% \rsSubject{пивет}
% \rsAdverbarial{привет}
% \rsPredicate{привет}

% \rsNoun{дом}
% \rsNoun[ед.ч.]{дом}
% % \rsNoun{дом}[ед.ч.]

% \rsPronoun{вашего}
% \rsVerb[инф.]{ходить}
% \rsUnion{и}



% #966842	(150,104,66)
% #f44747	(244,71,71)
% #eedc31	(238,220,49)
% #7fdb6a	(127,219,106)
% #0e68ce	(14,104,206)

\newcommand{\RSMlineWidth}{1.5}
\newcommand{\RSMsignXPad}{1.5}
\newcommand{\RSMsignYPad}{1.5}
\newcommand{\RSMsignHeight}{4}


\tikzset{
    RSMlineWidth/.style={
        line width=\RSMlineWidth pt,
    }
}



% \rsPostfixAux[color]{постфикс}
% \rsPostfixAux*{постфикс}


% \rsPostfixAux*[phantom]{постфикс}
% \rsPostfix{пост, фикс}
% \rsPostfix[phantom]{пост, фикс}
% \rsPostfix*{пост, фикс}
% \rsPostfix*[phantom]{пост, фикс}



% \rsRootAux{корень}
% \rsRootAux*[phantom]{корень}
% \rsRoot{кор, ень}
% \rsRoot[phantom]{кор, ень}
% \rsRoot*{кор, ень}
% \rsRoot*[phantom]{кор, ень}


% \rsSuffixAux{суффикс}
% \rsSuffixAux*[phantom]{суффикс}
% \rsSuffix{суф, фикс}
% \rsSuffix[phantom]{суф, фикс}
% \rsSuffix*{суф, фикс}
% \rsSuffix*[phantom]{суф, фикс}


% \begingroup
%     \renewcommand{\RSPhantomBool}{1}
%     \renewcommand{\RSColorBool}{1}

%     \rsPostfix{пост, фикс}
%     \rsRoot{кор, ень}
%     \rsPrefix{при, став, ка}
%     \rsSuffix{суф, фикс}
% \endgroup




% \begingroup
%     \renewcommand{\RSPhantomBool}{1}
%     \renewcommand{\RSColorBool}{1}

%     \rsPrefix{при, став, ка}
%     \rsRoot{кор, ень}
%     \rsSuffix{суф, фикс}
%     \rsPostfix{пост, фикс}

%     \renewcommand{\RSPhantomBool}{0}
%     \renewcommand{\RSColorBool}{0}
% \endgroup




\begin{rslingu}[color]
    \rsPrefix{при, став, ка}
    \rsRoot{кор, ень}
    \rsSuffix{суф, фикс}
    \rsPostfix{пост, фикс}
    \rsBase{основа}
    \rsEnding{}
    \rsMorphemicAnalysis{под, за}{голов}{ок}{}{}
\end{rslingu}

    
    \rsMorphemicAnalysis{}{караван}{}{}{}
    \rsMorphemicAnalysis{}{рубин}{}{}{}
 
    \rsMorphemicAnalysis[color, phantom]{под, за}{голов}{ок}{}{}

\end{document}