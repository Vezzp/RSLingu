\def\RSUlineYShift{2}
\def\RSUlineWidth{1.25}

\tikzset{
    RSUline/.style={
        line width=\RSUlineWidth pt,
    },
}


\definecolor{RSSubjectColor}{HTML}{673ab7}
\definecolor{RSPredicateColor}{HTML}{e81e62}
\definecolor{RSAttributeColor}{HTML}{2196f3}
\definecolor{RSAdverbarialColor}{HTML}{009688}
\definecolor{RSObjectColor}{HTML}{ffa500}


\ExplSyntaxOn
    \cs_generate_variant:Nn \seq_set_split:Nnn { NnV }

    \cs_new_protected:Nn \l__rs_syntax_choose_text:n {
        \group_begin:
            \def\l__do_spacing{0}
            \node[RSCage] (Root) {
                \clist_map_variable:nNn { #1 } {\l__item} {
                    \FPifeq{\l__do_spacing}{0}
                        \def\l__do_spacing{1}
                    \else
                        \space
                    \fi
                    \StrCut{\l__item}{=}\l__word\l__text
                    \seq_set_split:NnV \l__tmp_seq { + } { \l__text }
                    \bool_if:nTF { \RSPhantomBool || \l_rs_syntax_phantom_bool }
                    {
                        \textoverset{%
                            \tikz[baseline=(tmp.base)]{
                                \node[RSCage] (tmp) {\phantom{\l__word}};
                                \node[RSCage] at (tmp.center) {\textbullet};
                            }
                        }{ \seq_use:Nn \l__tmp_seq {,\space} }
                    }{
                        \textoverset{\l__word}{ \seq_use:Nn \l__tmp_seq {,\space} }
                    }
                    
                }\RSDownVPhantom
            };
        \group_end:
    }
    
    \cs_generate_variant:Nn \l__rs_syntax_choose_text:n { p }

\ExplSyntaxOff


% % ==============================================
% % =========== Subject (подлежащее) =============
% % ==============================================

%     % Underline subject.
%     \NewDocumentCommand{\rsUnderlineSubject}{ O{black} m }{%
%         \tikz[baseline=(Root.base)]{%
%             \node[RSCage] (Root) {\downVPhantom#2};
%             \draw[#1, RSUline]
%                 ([yshift=-\RSUlineYShift pt]Root.south west) --
%                 ([yshift=-\RSUlineYShift pt]Root.south east);
%         }%
%     }

%     % \NewDocumentCommand{\rsSubject}{ s m }{%
%     %     \rsUnderlineSubject
%     %         [\IfBooleanTF{#1}{RSSubjectColor}{}]
%     %         {\ProcessMap{\textoverset}{#2}[ ]}%
%     % }


% % ==============================================
% % =========== Predicate (сказуемое) ============
% % ==============================================

%     % Underline predicate.
%     \NewDocumentCommand{\rsUnderlinePredicate}{ O{black} m }{%
%         \tikz[baseline=(Root.base)]{%
%             \node[RSCage] (Root) {\downVPhantom#2};
%             \draw[#1, RSUline]
%                 ([yshift=-\RSUlineYShift pt]Root.south west) --
%                 ([yshift=-\RSUlineYShift pt]Root.south east);
%             \draw[#1, RSUline]
%                 ([yshift=-\RSUlineYShift - 2 pt]Root.south west) --
%                 ([yshift=-\RSUlineYShift - 2 pt]Root.south east);
%         }%
%     }

%     % \NewDocumentCommand{\rsPredicate}{ s m }{%
%     %     \rsUnderlinePredicate
%     %         [\IfBooleanTF{#1}{RSPredicateColor}{}]
%     %         {\ProcessMap{\textoverset}{#2}[ ]}%
%     % }


% % ==============================================
% % ======== Adverbarial (обстоятельство) ========
% % ==============================================

%     % Underline predicate.
%     \NewDocumentCommand{\rsUnderlineAdverbarial}{ O{black} m }{%
%         \tikz[baseline=(Root.base)]{%
%             \node[RSCage] (Root) {\downVPhantom#2};
%             \draw[
%                 #1,
%                 RSUline,
%                 dash pattern={on 5pt off 2pt on 1.5pt off 2pt},
%             ]
%                 ([yshift=-\RSUlineYShift pt]Root.south west) --
%                 ([yshift=-\RSUlineYShift pt]Root.south east);
%         }%
%     }

%     % \NewDocumentCommand{\rsAdverbarial}{ s m }{%
%     %     \rsUnderlineAdverbarial
%     %         [\IfBooleanTF{#1}{RSAdverbarialColor}{}]
%     %         {\ProcessMap{\textoverset}{#2}[ ]}%
%     % }
    
    
% % ==============================================
% % ======== Attribute (определение) =============
% % ==============================================

%     % Underline predicate.
%     \NewDocumentCommand{\rsUnderlineAttribute}{ O{black} m }{%
%         \tikz[baseline=(Root.base)]{%
%             \node[RSCage] (Root) {\downVPhantom#2};
%             \draw[
%                 #1,
%                 RSUline,
%                 decorate,
%                 decoration={
%                     complete sines,
%                     segment length=2.75pt,
%                     amplitude=1.75,
%                     mirror,
%                     start up,
%                     end down
%                 }
%             ]
%                 ([yshift=-\RSUlineYShift pt]Root.south west) --
%                 ([yshift=-\RSUlineYShift pt]Root.south east);
%         }%
%     }

%     % \NewDocumentCommand{\rsAttribute}{ s m }{%
%     %     \rsUnderlineAttribute
%     %         [\IfBooleanTF{#1}{RSAttributeColor}{}]
%     %         {\ProcessMap{\textoverset}{#2}[ ]}%
%     % }


% % ==============================================
% % ======== Adverbarial (обстоятельство) ========
% % ==============================================

%     % Underline predicate.
%     \NewDocumentCommand{\rsUnderlineObject}{ O{black} m }{%
%         \tikz[baseline=(Root.base)]{%
%             \node[RSCage] (Root) {\downVPhantom#2};
%             \draw[
%                 #1,
%                 RSUline,
%                 dash pattern={on 5pt off 2pt on 1pt off 0pt},
%             ]
%                 ([yshift=-\RSUlineYShift pt]Root.south west) --
%                 ([yshift=-\RSUlineYShift pt]Root.south east);
%         }%
%     }

%     % \NewDocumentCommand{\rsObject}{ s m }{%
%     %     \rsUnderlineObject
%     %         [\IfBooleanTF{#1}{RSObjectColor}{}]
%     %         {\ProcessMap{\textoverset}{#2}[ ]}%
%     % }



\NewDocumentCommand{\rsNoun}{ m O{} }{%
    \textoverset{#1}{сущ.\ifblank{#2}{}{, #2}}%
}


\NewDocumentCommand{\rsVerb}{ m O{} }{%
    \textoverset{#1}{глаг.\ifblank{#2}{}{, #2}}%
}


\NewDocumentCommand{\rsAdverb}{ m O{} }{%
    \textoverset{#1}{нареч.\ifblank{#2}{}{, #2}}%
}


\NewDocumentCommand{\rsPretext}{ m O{} }{%
    \textoverset{#1}{предлог\ifblank{#2}{}{, #2}}%
}


\NewDocumentCommand{\rsUnion}{ m O{} }{%
    \textoverset{#1}{союз\ifblank{#2}{}{, #2}}%
}

\NewDocumentCommand{\rsPronoun}{ m O{} }{%
    \textoverset{#1}{мест.\ifblank{#2}{}{, #2}}%
}

\NewDocumentCommand{\rsAdjective}{ m O{} }{%
    \textoverset{#1}{прил.\ifblank{#2}{}{, #2}}%
}

\NewDocumentCommand{\rsParticle}{ m O{} }{%
    \textoverset{#1}{част.\ifblank{#2}{}{, #2}}%
}