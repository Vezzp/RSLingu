\def\RSUlineYShift{2}
\def\RSUlineWidth{1.25}

\tikzset{
    RSUline/.style={
        line width=\RSUlineWidth pt,
        % inner sep=0pt,
        % outer sep=0pt,
        % overlay
    },
}


\definecolor{RSSubjectColor}{HTML}{673ab7}
\definecolor{RSPredicateColor}{HTML}{e81e62}
\definecolor{RSAttributeColor}{HTML}{2196f3}
\definecolor{RSAdverbarialColor}{HTML}{009688}
\definecolor{RSObjectColor}{HTML}{ffa500}


% ==============================================
% =========== Subject (подлежащее) =============
% ==============================================

    % Underline subject.
    \NewDocumentCommand{\rsUnderlineSubject}{ O{black} m }{%
        \tikz[baseline=(Root.base)]{%
            \node[RSCage] (Root) {\downVPhantom#2};
            \draw[#1, RSUline]
                ([yshift=-\RSUlineYShift pt]Root.south west) --
                ([yshift=-\RSUlineYShift pt]Root.south east);
        }%
    }

    \NewDocumentCommand{\rsSubject}{ s m }{%
        \rsUnderlineSubject
            [\IfBooleanTF{#1}{RSSubjectColor}{}]
            {\ProcessMap{\textoverset}{#2}[ ]}%
    }


% ==============================================
% =========== Predicate (сказуемое) ============
% ==============================================

    % Underline predicate.
    \NewDocumentCommand{\rsUnderlinePredicate}{ O{black} m }{%
        \tikz[baseline=(Root.base)]{%
            \node[RSCage] (Root) {\downVPhantom#2};
            \draw[#1, RSUline]
                ([yshift=-\RSUlineYShift pt]Root.south west) --
                ([yshift=-\RSUlineYShift pt]Root.south east);
            \draw[#1, RSUline]
                ([yshift=-\RSUlineYShift - 2 pt]Root.south west) --
                ([yshift=-\RSUlineYShift - 2 pt]Root.south east);
        }%
    }

    \NewDocumentCommand{\rsPredicate}{ s m }{%
        \rsUnderlinePredicate
            [\IfBooleanTF{#1}{RSPredicateColor}{}]
            {\ProcessMap{\textoverset}{#2}[ ]}%
    }


% ==============================================
% ======== Adverbarial (обстоятельство) ========
% ==============================================

    % Underline predicate.
    \NewDocumentCommand{\rsUnderlineAdverbarial}{ O{black} m }{%
        \tikz[baseline=(Root.base)]{%
            \node[RSCage] (Root) {\downVPhantom#2};
            \draw[
                #1,
                RSUline,
                dash pattern={on 5pt off 2pt on 1.5pt off 2pt},
            ]
                ([yshift=-\RSUlineYShift pt]Root.south west) --
                ([yshift=-\RSUlineYShift pt]Root.south east);
        }%
    }

    \NewDocumentCommand{\rsAdverbarial}{ s m }{%
        \rsUnderlineAdverbarial
            [\IfBooleanTF{#1}{RSAdverbarialColor}{}]
            {\ProcessMap{\textoverset}{#2}[ ]}%
    }
    
    
% ==============================================
% ======== Attribute (определение) =============
% ==============================================

    % Underline predicate.
    \NewDocumentCommand{\rsUnderlineAttribute}{ O{black} m }{%
        \tikz[baseline=(Root.base)]{%
            \node[RSCage] (Root) {\downVPhantom#2};
            \draw[
                #1,
                RSUline,
                decorate,
                decoration={
                    complete sines,
                    segment length=2.75pt,
                    amplitude=1.75,
                    mirror,
                    start up,
                    end down
                }
            ]
                ([yshift=-\RSUlineYShift pt]Root.south west) --
                ([yshift=-\RSUlineYShift pt]Root.south east);
        }%
    }

    \NewDocumentCommand{\rsAttribute}{ s m }{%
        \rsUnderlineAttribute
            [\IfBooleanTF{#1}{RSAttributeColor}{}]
            {\ProcessMap{\textoverset}{#2}[ ]}%
    }


% ==============================================
% ======== Adverbarial (обстоятельство) ========
% ==============================================

    % Underline predicate.
    \NewDocumentCommand{\rsUnderlineObject}{ O{black} m }{%
        \tikz[baseline=(Root.base)]{%
            \node[RSCage] (Root) {\downVPhantom#2};
            \draw[
                #1,
                RSUline,
                dash pattern={on 5pt off 2pt on 1pt off 0pt},
            ]
                ([yshift=-\RSUlineYShift pt]Root.south west) --
                ([yshift=-\RSUlineYShift pt]Root.south east);
        }%
    }

    \NewDocumentCommand{\rsObject}{ s m }{%
        \rsUnderlineObject
            [\IfBooleanTF{#1}{RSObjectColor}{}]
            {\ProcessMap{\textoverset}{#2}[ ]}%
    }



\NewDocumentCommand{\rsNoun}{ O{} m }{%
    \textoverset{#2}{сущ.\ifblank{#1}{}{, #1}}%
}


\NewDocumentCommand{\rsVerb}{ O{} m }{%
    \textoverset{#2}{глаг.\ifblank{#1}{}{, #1}}%
}


\NewDocumentCommand{\rsAdverb}{ O{} m }{%
    \textoverset{#2}{нареч.\ifblank{#1}{}{, #1}}%
}


\NewDocumentCommand{\rsPretext}{ O{} m }{%
    \textoverset{#2}{предлог\ifblank{#1}{}{, #1}}%
}


\NewDocumentCommand{\rsUnion}{ O{} m }{%
    \textoverset{#2}{союз\ifblank{#1}{}{, #1}}%
}

\NewDocumentCommand{\rsPronoun}{ O{} m }{%
    \textoverset{#2}{мест.\ifblank{#1}{}{, #1}}%
}

\NewDocumentCommand{\rsAdjective}{ O{} m }{%
    \textoverset{#2}{прил.\ifblank{#1}{}{, #1}}%
}

\NewDocumentCommand{\rsParticle}{ O{} m }{%
    \textoverset{#2}{част.\ifblank{#1}{}{, #1}}%
}