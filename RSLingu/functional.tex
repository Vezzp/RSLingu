% Process input of form `Key:Value`
\NewDocumentCommand{\ProcessPair}{ m m }{%
    \StrCut{#2}{:}\Key\Value%
    #1{\Key}{\Value}%
}


% Process input of form `Item1, Item2, ...`
\NewDocumentCommand{\ProcessVector}{ m m O{} }{%
    {%
        \newcommand{\isFirstIteration}{1}%
        \setsepchar{,}%
        \readlist*\looplist{#2}%
        \foreachitem \n \in \looplist{%
            \ifthenelse{\isFirstIteration = 1}{%
                \renewcommand{\isFirstIteration}{0}%
            }{#3}%
            #1{\n}%
        }%
    }%
}


% Process input of form `K1:V1, K2:V2, ...`.
\NewDocumentCommand{\ProcessMap}{ m m O{} }{%
    {%
        \newcommand{\tmpFn}[1]{\ProcessPair{#1}{##1}}%
        \ProcessVector{\tmpFn}{#2}[#3]%
    }%
}


% NOTE: Could be broaden to broader functionality.
\NewDocumentCommand{\textoverset}{ O{\commonYShift} m m }{%
    \tikz[baseline=(Root.base)]{%
        \node[inner sep=0pt,
              outer sep=0pt,
              label={[inner sep=0pt,
                      outer sep=0pt,
                      font=\tiny\itshape,
                      yshift=#1] #3\upperVPhantom}] (Root) {#2\upperVPhantom};%
    }%
}


% Extracts value from dimen.
% https://tex.stackexchange.com/questions/15001/getting-length-as-number
\newcommand*{\getlength}[2]{%
   \pgfmathsetmacro#1{#2}%  Result in `pt`
   % Or:
   %\pgfmathsetmacro#1{0.0351459804*#2}%  Result in `cm`
}


\newif\ifstartcompletesineup
\newif\ifendcompletesineup
\pgfkeys{
    /pgf/decoration/.cd,
    start up/.is if=startcompletesineup,
    start up=true,
    start up/.default=true,
    start down/.style={/pgf/decoration/start up=false},
    end up/.is if=endcompletesineup,
    end up=true,
    end up/.default=true,
    end down/.style={/pgf/decoration/end up=false}
}
\pgfdeclaredecoration{complete sines}{initial}
{
    \state{initial}[
        width=+0pt,
        next state=upsine,
        persistent precomputation={
            \ifstartcompletesineup
                \pgfkeys{/pgf/decoration automaton/next state=upsine}
                \ifendcompletesineup
                    \pgfmathsetmacro\matchinglength{
                        0.5*\pgfdecoratedinputsegmentlength / (ceil(0.5* \pgfdecoratedinputsegmentlength / \pgfdecorationsegmentlength) )
                    }
                \else
                    \pgfmathsetmacro\matchinglength{
                        0.5 * \pgfdecoratedinputsegmentlength / (ceil(0.5 * \pgfdecoratedinputsegmentlength / \pgfdecorationsegmentlength ) - 0.499)
                    }
                \fi
            \else
                \pgfkeys{/pgf/decoration automaton/next state=downsine}
                \ifendcompletesineup
                    \pgfmathsetmacro\matchinglength{
                        0.5* \pgfdecoratedinputsegmentlength / (ceil(0.5 * \pgfdecoratedinputsegmentlength / \pgfdecorationsegmentlength ) - 0.4999)
                    }
                \else
                    \pgfmathsetmacro\matchinglength{
                        0.5 * \pgfdecoratedinputsegmentlength / (ceil(0.5 * \pgfdecoratedinputsegmentlength / \pgfdecorationsegmentlength ) )
                    }
                \fi
            \fi
            \setlength{\pgfdecorationsegmentlength}{\matchinglength pt}
        }] {}
    \state{downsine}[width=\pgfdecorationsegmentlength,next state=upsine]{
        \pgfpathsine{\pgfpoint{0.5\pgfdecorationsegmentlength}{0.5\pgfdecorationsegmentamplitude}}
        \pgfpathcosine{\pgfpoint{0.5\pgfdecorationsegmentlength}{-0.5\pgfdecorationsegmentamplitude}}
    }
    \state{upsine}[width=\pgfdecorationsegmentlength,next state=downsine]{
        \pgfpathsine{\pgfpoint{0.5\pgfdecorationsegmentlength}{-0.5\pgfdecorationsegmentamplitude}}
        \pgfpathcosine{\pgfpoint{0.5\pgfdecorationsegmentlength}{0.5\pgfdecorationsegmentamplitude}}
}
    \state{final}{}
}


% https://tex.stackexchange.com/a/300215/192286
\ExplSyntaxOn
\DeclareExpandableDocumentCommand{\dorepeate}{O{}mm}
 {
      \int_compare:nT { #2 > 0 }
       {
            #3 \prg_replicate:nn { #2 - 1 } { #1#3 }
       }
 }
\ExplSyntaxOff



\NewDocumentCommand{\rsCodeAux}{ s m }{%
   \texttt{\IfBooleanTF{#1}{\textbackslash}\,#2}%
}

\NewDocumentCommand{\rsOptionsAux}{ s O{options} }{%
   \texttt{\IfBooleanTF{#1}{\textbackslash}<\textcolor{red!75}{#2}>}%
}

\NewDocumentCommand{\rsArgAux}{ s O{clist} O{} }{%
   \texttt{\IfBooleanTF{#1}{\textbackslash}<\textcolor{blue!75}{#2}>\textcolor{blue!75}{#3}}%
}

\NewDocumentCommand{\rsTypeAux}{ s O{cmd} }{%
   \texttt{\IfBooleanTF{#1}{\textbackslash}\textcolor{ForestGreen!75}{#2}}%
}


\NewDocumentCommand{\rsNameref}{ m }{%
    \ref{#1} <<\textit{\nameref{#1}}>>%
}