\newpage 

\thispagestyle{empty}

\vspace*{7cm}
\begin{center}
    { \huge \rsCodeAux{rslingu} } \\[1cm]
    { \Large \LaTeX-пакет для \rsMorphemicAnalysis[color]{}{школь}{н}{ой}{} \rsObject{лингвистики}}\\
\end{center}


\vspace*{7cm}
\begin{flushright}
    \begin{tabular}{@{}r@{\hskip .75cm}l@{}}
        Автор       &  Шлёнский Владислав \\
        E-mail      & \texttt{vladislav.shlenskii@yandex.ru} \\
        Обновлено   & 25.06.2020
    \end{tabular}
\end{flushright}


\newpage

\tableofcontents

\newpage


\section{Условные обозначения}

Формально описание команды или окружения будет даваться следующим образом, например:
\begin{tcolorbox}
    \small \rsTypeAux:\rsCodeAux{rsPrefix[\rsOptionsAux]\{\rsArgAux\}} \\
    \hspace*{1cm} \rsOptionsAux: \rsCodeAux{color:bool=false, phantom:bool=false}. 
\end{tcolorbox}

Разберём первую строчку, которая содержит сигнатуру объекта-модификатора текста:
\begin{center}
    \rsTypeAux:\rsCodeAux{rsPrefix[\rsOptionsAux]\{\rsArgAux\}}
\end{center}
Сначала идёт обозначение типа объекта-модификатора текста, он может быть либо \rsTypeAux\space --- для команды --- или \rsTypeAux[env] --- для окружения. Далее идёт символ двоеточия <<\rsCodeAux{:}>>, смысловой нагрузки он несёт --- это просто разделитель для лучшей читаемости. Затем имя, как в данном примере, команды --- \rsCodeAux{rsPrefix}. 

После следуют аргументы, которые принимает команда или окружение. Обязательные аргументы пишутся в фигурных скобках; необязательные --- в квадратных. В каждой из скобок указан ожидаемый вход, который может принимать несколько значений:
\begin{itemize}
    \item \rsOptionsAux. Означает, что в данное место передать дополнительные параметры, которые меняют поведение команды по умолчанию. То, какие дополнительные параметры можно передавать, а так же значения по умолчанию, описывается на следующих строчках.
    \item \rsArgAux[tl]\space (от англ. \textit{token list}). Означает, что весь передаваемый текст будет обработан целиком.
    \item \rsArgAux\space (от англ. \textit{comma list}). Означает, что при наличии в передаваемом тексте запятых, каждая часть переданной строки, находящаяся между запятыми (либо началом строчки и запятой или запятой и концом строчки), будет обработана отдельно.
    \item В некоторых командах, предназначенных, например, для синтаксического разбора предложений, происходит, вообще говоря, разбиение, или парсинг (от англ. \textit{parse}), входной строчки по нескольким разделителям. Например, если парсинг возможен сначала по запятой, потом по знаку <<\texttt{=}>> и, наконец, скажем, по знаку <<\texttt{+}>>, это будет указано следующим образом:\footnote{Для наглядной демонстрации написанного смотрите пункт \rsNameref{sec:SyntaxAnalysis}.}
    \begin{center}
        \rsArgAux[clist][[=, +]].
    \end{center}
\end{itemize}


Разберём теперь вторую строчку, в которой разъясняются дополнительные параметры:
\begin{center}
    \rsOptionsAux: \rsCodeAux{color:bool=false, phantom:bool=false}. 
\end{center}

После двоеточия следует перечисление через запятую возможных параметров для данного объекта, их тип и значение по умолчанию.



\section{Морфемный анализ слов} 

\subsection{Приставка}

\begin{tcolorbox}
    \small
    \rsTypeAux:\rsCodeAux{rsPrefix[\rsOptionsAux]\{\rsArgAux\}} \\
    \hspace*{1cm} \rsOptionsAux: \rsCodeAux{color:bool=false, phantom:bool=false}. 
\end{tcolorbox}

\begingroup
\renewcommand{\arraystretch}{1.125}
\begin{table}[ht!]
    \centering
    \begin{tabular}{|l|l|}
        \hline
        \rsCodeAux*{rsPrefix{\{\}}} & \rsPrefix{} \\
        \rsCodeAux*{rsPrefix{\{приставка\}}} & \rsPrefix{приставка} \\
        \rsCodeAux*{rsPrefix{\{при, став, ка\}}} & \rsPrefix{при, став, ка} \\
        \rsCodeAux*{rsPrefix[color]{\{при, став, ка\}}} & \rsPrefix[color]{при, став, ка} \\
        \rsCodeAux*{rsPrefix[phantom]{\{при, став, ка\}}} & \rsPrefix[phantom]{при, став, ка} \\
        \rsCodeAux*{rsPrefix[color, phantom]{\{при, став, ка\}}} & \rsPrefix[color, phantom]{при, став, ка} \\
        \hline
    \end{tabular}
    \caption{Использование команды приставки.}
\end{table}
\endgroup




\subsection{Корень}

\begin{tcolorbox}
    \small
    \rsTypeAux:\rsCodeAux{rsRoot[\rsOptionsAux]\{\rsArgAux\}} \\
    \hspace*{1cm} \rsOptionsAux: \rsCodeAux{color:bool=false, phantom:bool=false}.
\end{tcolorbox}

\begingroup
\renewcommand{\arraystretch}{1.125}
\begin{table}[ht!]
    \centering
    \begin{tabular}{|l|l|}
        \hline
        \rsCodeAux*{rsRoot{\{\}}} & \rsRoot{} \\
        \rsCodeAux*{rsRoot{\{корень\}}} & \rsRoot{корень} \\
        \rsCodeAux*{rsRoot{\{кор, ень\}}} & \rsRoot{кор, ень} \\
        \rsCodeAux*{rsRoot[color]{\{кор, ень\}}} & \rsRoot[color]{кор, ень} \\
        \rsCodeAux*{rsRoot[phantom]{\{кор, ень\}}} & \rsRoot[phantom]{кор, ень} \\
        \rsCodeAux*{rsRoot[color, phantom]{\{кор, ень\}}} & \rsRoot[color, phantom]{кор, ень} \\
        \hline
    \end{tabular}
    \caption{Использование команды корня.}
\end{table} 
\endgroup




\subsection{Суффикс}
    
\begin{tcolorbox}
    \small
    \rsTypeAux:\rsCodeAux{rsSuffix[\rsOptionsAux]\{\rsArgAux\}} \\
    \hspace*{1cm} \rsOptionsAux: \rsCodeAux{color:bool=false, phantom:bool=false}.
\end{tcolorbox}    

\begingroup
\renewcommand{\arraystretch}{1.125}
\begin{table}[ht!]
    \centering
    \begin{tabular}{|l|l|}
        \hline
        \rsCodeAux*{rsSuffix{\{\}}} & \rsSuffix{} \\
        \rsCodeAux*{rsSuffix{\{суффикс\}}} & \rsSuffix{суффикс} \\
        \rsCodeAux*{rsSuffix{\{суф, фикс\}}} & \rsSuffix{суф, фикс} \\
        \rsCodeAux*{rsSuffix[color]{\{суф, фикс\}}} & \rsSuffix[color]{суф, фикс} \\
        \rsCodeAux*{rsSuffix[phantom]{\{суф, фикс\}}} & \rsSuffix[phantom]{суф, фикс} \\
        \rsCodeAux*{rsSuffix[color, phantom]{\{суф, фикс\}}} & \rsSuffix[color, phantom]{суф, фикс} \\
        \hline
    \end{tabular}
    \caption{Использование команды суффикса.}
\end{table}
\endgroup




\subsection{Окончание}

\begin{tcolorbox}
    \small
    \rsTypeAux:\rsCodeAux{rsEnding[\rsOptionsAux]\{\rsArgAux[tl]\}} \\
    \hspace*{1cm} \rsOptionsAux: \rsCodeAux{color:bool=false, phantom:bool=false}.
\end{tcolorbox}

\begingroup
\renewcommand{\arraystretch}{1.125}
\begin{table}[ht!]
    \centering
    \begin{tabular}{|l|l|}
        \hline
        \rsCodeAux*{rsEnding{\{\}}} & \rsEnding{} \\
        \rsCodeAux*{rsEnding{\{окончание\}}} & \rsEnding{окончание} \\
        \rsCodeAux*{rsEnding[color]{\{окончание\}}} & \rsEnding[color]{окончание} \\
        \rsCodeAux*{rsEnding[phantom]{\{окончание\}}} & \rsEnding[phantom]{окончание} \\
        \rsCodeAux*{rsEnding[color, phantom]{\{окончание\}}} & \rsEnding[color, phantom]{окончание} \\
        \hline
    \end{tabular}
    \caption{Использование команды окончания.}
\end{table}
\endgroup




\subsection{Постфикc}

\begin{tcolorbox}
    \small
    \rsTypeAux:\rsCodeAux{rsPostfix[\rsOptionsAux]\{\rsArgAux\}} \\
    \hspace*{1cm} \rsOptionsAux: \rsCodeAux{color:bool=false, phantom:bool=false}.
\end{tcolorbox}

\begingroup
\renewcommand{\arraystretch}{1.125}
\begin{table}[ht!]
    \centering
    \begin{tabular}{|l|l|}
        \hline
        \rsCodeAux*{rsPostfix{\{\}}} & \rsPostfix{} \\
        \rsCodeAux*{rsPostfix{\{постфикс\}}} & \rsPostfix{постфикс} \\
        \rsCodeAux*{rsPostfix{\{пост, фикс\}}} & \rsPostfix{пост, фикс} \\
        \rsCodeAux*{rsPostfix[color]{\{пост, фикс\}}} & \rsPostfix[color]{пост, фикс} \\
        \rsCodeAux*{rsPostfix[phantom]{\{пост, фикс\}}} & \rsPostfix[phantom]{пост, фикс} \\
        \rsCodeAux*{rsPostfix[color, phantom]{\{пост, фикс\}}} & \rsPostfix[color, phantom]{пост, фикс} \\
        \hline
    \end{tabular}
    \caption{Использование команды постфикса.}
\end{table}
\endgroup




\subsection{Основа}

\begin{tcolorbox}
    \small
    \rsTypeAux:\rsCodeAux{rsBase[\rsOptionsAux]\{\rsArgAux[tl]\}} \\
    \hspace*{1cm} \rsOptionsAux: \rsCodeAux{color:bool=false, right:bool=false, left:bool=false}.
\end{tcolorbox}

\begingroup
\renewcommand{\arraystretch}{1.125}
\begin{table}[ht!]
    \centering
    \begin{tabular}{|l|l|}
        \hline
        \rsCodeAux*{rsBase{\{основа\}}} & \rsBase{основа} \\
        \rsCodeAux*{rsBase[color]{\{основа\}}} & \rsBase[color]{основа} \\
        \rsCodeAux*{rsBase[left]{\{основа\}}} & \rsBase[left]{основа} \\
        \rsCodeAux*{rsBase[color, right]{\{основа\}}} & \rsBase[color, right]{основа} \\
        \hline
    \end{tabular}
    \caption{Использование команды основы.}
\end{table}
\endgroup




\subsection{Разбор слова с непрерывной основной}

\begin{tcolorbox}
    \small
    \rsTypeAux:\rsCodeAux{rsMorphemicAnalysis[\rsOptionsAux]\{\rsArgAux\}\{\rsArgAux\}\{\rsArgAux\}\{\rsArgAux[tl]\}\{\rsArgAux\}} \\
    \hspace*{1cm} \rsOptionsAux: \rsCodeAux{color:bool=false, phantom:bool=false}.
\end{tcolorbox}    

\begingroup
\renewcommand{\arraystretch}{1.125}
\begin{table}[ht!]
    \centering
    \begin{tabular}{|l|l|}
        \hline
        {\small \rsCodeAux*{rsMorphemicAnalysis\{бес, при\}\{дан\}\{н, ниц\}\{а\}\{\}}} & \rsMorphemicAnalysis{бес, при}{дан}{н, ниц}{а}{} \\
        {\small \rsCodeAux*{rsMorphemicAnalysis\{из\}\{маз\}\{а, л\}\{\}\{ся\}}} & \rsMorphemicAnalysis{из}{маз}{а, л}{}{ся} \\
        {\small \rsCodeAux*{rsMorphemicAnalysis[phantom]\{из\}\{маз\}\{а, л\}\{\}\{ся\}}} & \rsMorphemicAnalysis[phantom]{из}{маз}{а, л}{}{ся} \\
        {\small \rsCodeAux*{rsMorphemicAnalysis[color]\{вне\}\{штат\}\{н\}\{ый\}\{\}}} & \rsMorphemicAnalysis[color]{вне}{штат}{н}{ый}{} \\
        \hline
    \end{tabular}
    \caption{Использование команды разбора слова.}
\end{table}
\endgroup




\section{Части речи}

Все команды данной группы имеют вид:
\begin{tcolorbox}
    \small
    \rsTypeAux:\rsCodeAux{<name>\{\rsArgAux[tl]\}[\rsArgAux[tl]]}
\end{tcolorbox}
где \rsCodeAux{<name>} может принимать одно из следующих значений:
\begin{table}[ht!]
    \centering
    \begin{tabular}{|l|l|l|l|}
        \hline
        \rsCodeAux{rsNoun} & существительное & \rsCodeAux{rsVerb} & глагол \\\hline
        \rsCodeAux{rsAdverb} & наречие & \rsCodeAux{rsPretext} & предлог \\\hline
        \rsCodeAux{rsUnion} & союз & \rsCodeAux{rsPronoun} & местоимение \\\hline
        \rsCodeAux{rsAdjective} & прилагательное & \rsCodeAux{rsParticle} & частица \\\hline
    \end{tabular}
    \caption{Команды для частей речи.}
\end{table}

\begin{table}[ht!]
    \centering
    \begin{tabular}{|l|l|}
        \hline
        \rsCodeAux*{rsNoun\{существительное\}} & \rsNoun{существительное} \\ 
        \rsCodeAux*{rsNoun\{существительное\}[ср.р., им.п.]} & \rsNoun{существительное}[ср.р., им.п.] \\
        \rsCodeAux*{rsVerb\{глагол\}} & \rsVerb{глагол} \\
        \rsCodeAux*{rsAdverb\{наречие\}} & \rsAdverb{наречие} \\
        \rsCodeAux*{rsPretext\{предлог\}} & \rsAdverb{предлог} \\
        \rsCodeAux*{rsUnion\{союз\}} & \rsUnion{союз} \\
        \rsCodeAux*{rsPronoun\{местоимение\}} & \rsUnion{местоимение} \\
        \rsCodeAux*{rsAdjective\{прилагательное\}} & \rsAdjective{прилагательное} \\
        \rsCodeAux*{rsParticle\{частица\}} & \rsParticle{частица} \\\hline
    \end{tabular}
    \caption{Использование команд для частей речи.}
\end{table}



% ======================================================
% ======== Синтаксический разбор предложений ===========
% ======================================================

\section{Синтаксический разбор предложений}
\label{sec:SyntaxAnalysis}

\subsection{Подлежащее}
\begin{tcolorbox}
    \rsTypeAux:\rsCodeAux{rsSubject[\rsOptionsAux]\{\rsArgAux[clist][[=, +]]\}} \\
    \hspace*{1cm} \rsOptionsAux: \rsCodeAux{color:bool=false, phantom:bool=false}.
\end{tcolorbox}


\begingroup
\renewcommand{\arraystretch}{1.125}
\begin{table}[ht!]
    \centering
    \small
    \begin{tabular}{|l|l|}
        \hline
        \rsCodeAux*{rsSubject{\{подлежащее\}}} & \rsSubject{подлежащее} \\
        \rsCodeAux*{rsSubject[color]{\{подлежащее\}}} & \rsSubject[color]{подлежащее} \\
        \rsCodeAux*{rsSubject[phantom, color]{\{подлежащее\}}} & \rsSubject[phantom, color]{подлежащее} \\
        \rsCodeAux*{rsSubject[phantom, color]{\{подлежащее=сущ.\}}} & \rsSubject[color]{подлежащее=сущ.} \\
        \rsCodeAux*{rsSubject{\{подлежащее, подлежащее\}}} & \rsSubject{подлежащее, подлежащее} \\
        \rsCodeAux*{rsSubject{\{подлежащее=сущ. + им.п + ср.р., подлежащее=ср.р.\}}} & \rsSubject{подлежащее=сущ. + им.п. + ср.р., подлежащее=ср.р} \\
        \rsCodeAux*{rsSubject[color, phantom]{\{подлежащее, подлежащее=ср.р.\}}} & \rsSubject[color, phantom]{подлежащее, подлежащее=ср.р.} \\
        \hline
    \end{tabular}
    \caption{Использование команды подлежащего.}
\end{table}
\endgroup


\subsection{Сказуемое}
\begin{tcolorbox}
    \rsTypeAux:\rsCodeAux{rsPredicate[\rsOptionsAux]\{\rsArgAux[clist][[=, +]]\}} \\
    \hspace*{1cm} \rsOptionsAux: \rsCodeAux{color:bool=false, phantom:bool=false}.
\end{tcolorbox}

\begingroup
\renewcommand{\arraystretch}{1.125}
\begin{table}[ht!]
    \centering
    \small
    \begin{tabular}{|l|l|}
        \hline
        \rsCodeAux*{rsPredicate{\{сказуемое\}}} & \rsPredicate{сказуемое} \\
        \rsCodeAux*{rsPredicate[color]{\{сказуемое\}}} & \rsPredicate[color]{сказуемое} \\
        \rsCodeAux*{rsPredicate[phantom, color]{\{сказуемое\}}} & \rsPredicate[phantom, color]{сказуемое} \\
        \rsCodeAux*{rsPredicate[phantom, color]{\{сказуемое=глаг.\}}} & \rsPredicate[color]{сказуемое=глаг.} \\
        \rsCodeAux*{rsPredicate{\{сказуемое, сказуемое\}}} & \rsPredicate{сказуемое, сказуемое} \\
        \rsCodeAux*{rsPredicate{\{сказуемое=глаг. + н.в., сказуемое=пр.в.\}}} & \rsPredicate{сказуемое=глаг. + н.в. + пр.в., сказуемое=б.в.} \\
        \rsCodeAux*{rsPredicate[color, phantom]{\{сказуемое, сказуемое=пр.в.\}}} & \rsPredicate[color, phantom]{сказуемое, сказуемое=пр.в.} \\
        \hline
    \end{tabular}
    \caption{Использование команды сказуемого.}
\end{table}
\endgroup


\subsection{Дополнение}
\begin{tcolorbox}
    \rsTypeAux:\rsCodeAux{rsObject[\rsOptionsAux]\{\rsArgAux[clist][[=, +]]\}} \\
    \hspace*{1cm} \rsOptionsAux: \rsCodeAux{color:bool=false, phantom:bool=false}.
\end{tcolorbox}

\begingroup
\renewcommand{\arraystretch}{1.125}
\begin{table}[ht!]
    \centering
    \small
    \begin{tabular}{|l|l|}
        \hline
        \rsCodeAux*{rsObject{\{дополнение\}}} & \rsObject{дополнение} \\
        \rsCodeAux*{rsObject[color]{\{дополнение\}}} & \rsObject[color]{дополнение} \\
        \rsCodeAux*{rsObject[phantom, color]{\{дополнение\}}} & \rsObject[phantom, color]{дополнение} \\
        \rsCodeAux*{rsObject[phantom, color]{\{дополнение=сущ.\}}} & \rsObject[color]{дополнение=сущ.} \\
        \rsCodeAux*{rsObject{\{дополнение, дополнение\}}} & \rsObject{дополнение, дополнение} \\
        \rsCodeAux*{rsObject{\{дополнение=сущ., дополнение=сущ. + им.п.\}}} & \rsObject{дополнение=сущ., дополнение=сущ. + им.п.} \\
        \rsCodeAux*{rsObject[color, phantom]{\{дополнение, дополнение=сущ.\}}} & \rsObject[color, phantom]{дополнение, дополнение=сущ.} \\
        \hline
    \end{tabular}
    \caption{Использование команды дополнения.}
\end{table}
\endgroup



\subsection{Определение}
\begin{tcolorbox}
    \rsTypeAux:\rsCodeAux{rsAttribute[\rsOptionsAux]\{\rsArgAux[clist][[=, +]]\}} \\
    \hspace*{1cm} \rsOptionsAux: \rsCodeAux{color:bool=false, phantom:bool=false}.
\end{tcolorbox}

\begingroup
\renewcommand{\arraystretch}{1.125}
\begin{table}[ht!]
    \centering
    \small
    \begin{tabular}{|l|l|}
        \hline
        \rsCodeAux*{rsAttribute{\{определение\}}} & \rsAttribute{определение} \\
        \rsCodeAux*{rsAttribute[color]{\{определение\}}} & \rsAttribute[color]{определение} \\
        \rsCodeAux*{rsAttribute[phantom, color]{\{определение\}}} & \rsAttribute[phantom, color]{определение} \\
        \rsCodeAux*{rsAttribute[phantom, color]{\{определение=прил.\}}} & \rsAttribute[color]{определение=прил.} \\
        \rsCodeAux*{rsAttribute{\{определение, определение\}}} & \rsAttribute{определение, определение} \\
        \rsCodeAux*{rsAttribute{\{определение=прил., определение=прил. + согл.\}}} & \rsAttribute{определение=прил., определение=прил. + согл.} \\
        \rsCodeAux*{rsAttribute[color, phantom]{\{определение=прил., определение\}}} & \rsAttribute[color, phantom]{определение=прил., определение} \\
        \hline
    \end{tabular}
    \caption{Использование команды определения.}
\end{table}
\endgroup


\subsection{Обстоятельство}
\begin{tcolorbox}
    \rsTypeAux:\rsCodeAux{rsAdverbarial[\rsOptionsAux]\{\rsArgAux[clist][[=, +]]\}} \\
    \hspace*{1cm} \rsOptionsAux: \rsCodeAux{color:bool=false, phantom:bool=false}.
\end{tcolorbox}


\begingroup
\renewcommand{\arraystretch}{1.125}
\begin{table}[ht!]
    \centering
    \small
    \begin{tabular}{|l|l|}
        \hline
        \rsCodeAux*{rsAdverbarial{\{обстоятельство\}}} & \rsAdverbarial{обстоятельство} \\
        \rsCodeAux*{rsAdverbarial[color]{\{обстоятельство\}}} & \rsAdverbarial[color]{обстоятельство} \\
        \rsCodeAux*{rsAdverbarial[phantom, color]{\{обстоятельство\}}} & \rsAdverbarial[phantom, color]{обстоятельство} \\
        \rsCodeAux*{rsAdverbarial[phantom, color]{\{обстоятельство=сущ. + им.п.\}}} & \rsAdverbarial[color]{обстоятельство=сущ. + им.п.} \\
        \hline
    \end{tabular}
    \caption{Использование команды обстоятельства.}
\end{table}
\endgroup



\section{Прочее}

\subsection{Окружение rslingu}

\begin{tcolorbox}
    \noindent\rsTypeAux[env]:\rsCodeAux{rslingu[\rsOptionsAux]} \\
    \hspace*{1cm} \rsOptionsAux: \rsCodeAux{color:bool=false, phantom:bool=false}.
\end{tcolorbox}

Иногда может возникать необходимость, например, морфемного разбора слов с <<разрывной>> основой. Для таких случаев нет специально
определённых команд, подобно команде \rsCodeAux*{rsMorphemicAnalysis}, так что единственный способ отобразить такие слова --- это
последовательное использование команд для каждой из морфем. При передаче параметров \rsCodeAux{phantom} и \rsCodeAux{color} в каждую из
команд возникает многократное дублирование кода, что ухудшает его читаемость.



Решить эту проблему призвано окружение \rsCodeAux{rslingu}, которое указании какого-либо дополнительного аргумента, <<активирует>> его для всех команд, принимающий данный аргумент, внутри окружения.
\begin{figure}[htp!]
    \centering
    \begin{subfigure}{\textwidth}
        \begin{minted}[
            frame=lines,
            framesep=2mm,
            baselinestretch=1.2,
            bgcolor=black!5!white,
            fontsize=\small,
            linenos
        ]{latex}
\begin{rslingu}[color]
    \rsAttribute{Уставшая} \rsSubject{мама} \rsPredicate{мыла} \rsObject{раму}
    \rsAdverbarial{вечером}.
\end{rslingu}
        \end{minted}
    \caption{Код с использованием окружения \rsCodeAux{rslingu}.}
    \end{subfigure}\vspace*{.75cm}
    \begin{subfigure}{\textwidth}
        \begin{minted}[
            frame=lines,
            framesep=2mm,
            baselinestretch=1.2,
            bgcolor=black!5!white,
            fontsize=\small,
            linenos
        ]{latex}
\rsAttribute[color]{Уставшая} \rsSubject[color]{мама} \rsPredicate[color]{мыла}
\rsObject[color]{раму} \rsAdverbarial[color]{вечером}.
        \end{minted}
    \caption{Код без использования окружения \rsCodeAux{rslingu}.}
    \end{subfigure}\vspace*{.75cm}
    \begin{subfigure}{.9\textwidth}
        \centering
        \begin{rslingu}[color]
            \rsAttribute{Уставшая} \rsSubject{мама} \rsPredicate{мыла} \rsObject{раму} \rsAdverbarial{вечером}.
        \end{rslingu}
    \caption{Результат выполнения каждого из частей кода выше.}
    \end{subfigure}
    \caption{Демонстрация возможностей окружения \rsCodeAux{rslingu} c параметром \rsCodeAux{color}.}
\end{figure}



\begin{figure}[htp!]
    \centering
    \begin{subfigure}{\textwidth}
        \begin{minted}[
            frame=lines,
            framesep=2mm,
            baselinestretch=1.2,
            bgcolor=black!5!white,
            fontsize=\small,
            linenos
        ]{latex}
\begin{rslingu}[color, phantom]
    \rsAttribute{Уставшая} \rsSubject{мама} \rsPredicate{мыла} \rsObject{раму}
    \rsAdverbarial{вечером}.
\end{rslingu}
        \end{minted}
    \caption{Код с использованием окружения \rsCodeAux{rslingu}.}
    \end{subfigure}\vspace*{.75cm}
    \begin{subfigure}{\textwidth}
        \begin{minted}[
            frame=lines,
            framesep=2mm,
            baselinestretch=1.2,
            bgcolor=black!5!white,
            fontsize=\small,
            linenos
        ]{latex}
\rsAttribute[color, phantom]{Уставшая} \rsSubject[phantom, color]{мама}
\rsPredicate[phantom, color]{мыла} \rsObject[phantom, color]{раму}
\rsAdverbarial[phantom, color]{вечером}.
        \end{minted}
    \caption{Код без использования окружения \rsCodeAux{rslingu}.}
    \end{subfigure}\vspace*{.75cm}
    \begin{subfigure}{.9\textwidth}
        \centering
        \begin{rslingu}[color, phantom]
            \rsAttribute{Уставшая} \rsSubject{мама} \rsPredicate{мыла} \rsObject{раму} \rsAdverbarial{вечером}.
        \end{rslingu}
    \caption{Результат выполнения каждого из частей кода выше.}
    \end{subfigure}
    \caption{Демонстрация возможностей окружения \rsCodeAux{rslingu} c параметрами \rsCodeAux{color} и \rsCodeAux{phantom}.}
\end{figure}
